% Options for packages loaded elsewhere
\PassOptionsToPackage{unicode}{hyperref}
\PassOptionsToPackage{hyphens}{url}
%
\documentclass[
  a4paper,
  oneside,
  openany,
  12pt,
  onecolumn,
  twoside]{book}

\usepackage{amsmath,amssymb}
\usepackage{setspace}
\usepackage{iftex}
\ifPDFTeX
  \usepackage[T1]{fontenc}
  \usepackage[utf8]{inputenc}
  \usepackage{textcomp} % provide euro and other symbols
\else % if luatex or xetex
  \usepackage{unicode-math}
  \defaultfontfeatures{Scale=MatchLowercase}
  \defaultfontfeatures[\rmfamily]{Ligatures=TeX,Scale=1}
\fi
\usepackage{lmodern}
\ifPDFTeX\else  
    % xetex/luatex font selection
    \setmainfont[]{Public Sans}
\fi
% Use upquote if available, for straight quotes in verbatim environments
\IfFileExists{upquote.sty}{\usepackage{upquote}}{}
\IfFileExists{microtype.sty}{% use microtype if available
  \usepackage[]{microtype}
  \UseMicrotypeSet[protrusion]{basicmath} % disable protrusion for tt fonts
}{}
\makeatletter
\@ifundefined{KOMAClassName}{% if non-KOMA class
  \IfFileExists{parskip.sty}{%
    \usepackage{parskip}
  }{% else
    \setlength{\parindent}{0pt}
    \setlength{\parskip}{6pt plus 2pt minus 1pt}}
}{% if KOMA class
  \KOMAoptions{parskip=half}}
\makeatother
\usepackage{xcolor}
\usepackage[top=30mm,left=25mm,right=25mm,bottom=30mm]{geometry}
\setlength{\emergencystretch}{3em} % prevent overfull lines
\setcounter{secnumdepth}{2}
% Make \paragraph and \subparagraph free-standing
\makeatletter
\ifx\paragraph\undefined\else
  \let\oldparagraph\paragraph
  \renewcommand{\paragraph}{
    \@ifstar
      \xxxParagraphStar
      \xxxParagraphNoStar
  }
  \newcommand{\xxxParagraphStar}[1]{\oldparagraph*{#1}\mbox{}}
  \newcommand{\xxxParagraphNoStar}[1]{\oldparagraph{#1}\mbox{}}
\fi
\ifx\subparagraph\undefined\else
  \let\oldsubparagraph\subparagraph
  \renewcommand{\subparagraph}{
    \@ifstar
      \xxxSubParagraphStar
      \xxxSubParagraphNoStar
  }
  \newcommand{\xxxSubParagraphStar}[1]{\oldsubparagraph*{#1}\mbox{}}
  \newcommand{\xxxSubParagraphNoStar}[1]{\oldsubparagraph{#1}\mbox{}}
\fi
\makeatother

\usepackage{color}
\usepackage{fancyvrb}
\newcommand{\VerbBar}{|}
\newcommand{\VERB}{\Verb[commandchars=\\\{\}]}
\DefineVerbatimEnvironment{Highlighting}{Verbatim}{commandchars=\\\{\}}
% Add ',fontsize=\small' for more characters per line
\usepackage{framed}
\definecolor{shadecolor}{RGB}{241,243,245}
\newenvironment{Shaded}{\begin{snugshade}}{\end{snugshade}}
\newcommand{\AlertTok}[1]{\textcolor[rgb]{0.68,0.00,0.00}{#1}}
\newcommand{\AnnotationTok}[1]{\textcolor[rgb]{0.37,0.37,0.37}{#1}}
\newcommand{\AttributeTok}[1]{\textcolor[rgb]{0.40,0.45,0.13}{#1}}
\newcommand{\BaseNTok}[1]{\textcolor[rgb]{0.68,0.00,0.00}{#1}}
\newcommand{\BuiltInTok}[1]{\textcolor[rgb]{0.00,0.23,0.31}{#1}}
\newcommand{\CharTok}[1]{\textcolor[rgb]{0.13,0.47,0.30}{#1}}
\newcommand{\CommentTok}[1]{\textcolor[rgb]{0.37,0.37,0.37}{#1}}
\newcommand{\CommentVarTok}[1]{\textcolor[rgb]{0.37,0.37,0.37}{\textit{#1}}}
\newcommand{\ConstantTok}[1]{\textcolor[rgb]{0.56,0.35,0.01}{#1}}
\newcommand{\ControlFlowTok}[1]{\textcolor[rgb]{0.00,0.23,0.31}{\textbf{#1}}}
\newcommand{\DataTypeTok}[1]{\textcolor[rgb]{0.68,0.00,0.00}{#1}}
\newcommand{\DecValTok}[1]{\textcolor[rgb]{0.68,0.00,0.00}{#1}}
\newcommand{\DocumentationTok}[1]{\textcolor[rgb]{0.37,0.37,0.37}{\textit{#1}}}
\newcommand{\ErrorTok}[1]{\textcolor[rgb]{0.68,0.00,0.00}{#1}}
\newcommand{\ExtensionTok}[1]{\textcolor[rgb]{0.00,0.23,0.31}{#1}}
\newcommand{\FloatTok}[1]{\textcolor[rgb]{0.68,0.00,0.00}{#1}}
\newcommand{\FunctionTok}[1]{\textcolor[rgb]{0.28,0.35,0.67}{#1}}
\newcommand{\ImportTok}[1]{\textcolor[rgb]{0.00,0.46,0.62}{#1}}
\newcommand{\InformationTok}[1]{\textcolor[rgb]{0.37,0.37,0.37}{#1}}
\newcommand{\KeywordTok}[1]{\textcolor[rgb]{0.00,0.23,0.31}{\textbf{#1}}}
\newcommand{\NormalTok}[1]{\textcolor[rgb]{0.00,0.23,0.31}{#1}}
\newcommand{\OperatorTok}[1]{\textcolor[rgb]{0.37,0.37,0.37}{#1}}
\newcommand{\OtherTok}[1]{\textcolor[rgb]{0.00,0.23,0.31}{#1}}
\newcommand{\PreprocessorTok}[1]{\textcolor[rgb]{0.68,0.00,0.00}{#1}}
\newcommand{\RegionMarkerTok}[1]{\textcolor[rgb]{0.00,0.23,0.31}{#1}}
\newcommand{\SpecialCharTok}[1]{\textcolor[rgb]{0.37,0.37,0.37}{#1}}
\newcommand{\SpecialStringTok}[1]{\textcolor[rgb]{0.13,0.47,0.30}{#1}}
\newcommand{\StringTok}[1]{\textcolor[rgb]{0.13,0.47,0.30}{#1}}
\newcommand{\VariableTok}[1]{\textcolor[rgb]{0.07,0.07,0.07}{#1}}
\newcommand{\VerbatimStringTok}[1]{\textcolor[rgb]{0.13,0.47,0.30}{#1}}
\newcommand{\WarningTok}[1]{\textcolor[rgb]{0.37,0.37,0.37}{\textit{#1}}}

\providecommand{\tightlist}{%
  \setlength{\itemsep}{0pt}\setlength{\parskip}{0pt}}\usepackage{longtable,booktabs,array}
\usepackage{calc} % for calculating minipage widths
% Correct order of tables after \paragraph or \subparagraph
\usepackage{etoolbox}
\makeatletter
\patchcmd\longtable{\par}{\if@noskipsec\mbox{}\fi\par}{}{}
\makeatother
% Allow footnotes in longtable head/foot
\IfFileExists{footnotehyper.sty}{\usepackage{footnotehyper}}{\usepackage{footnote}}
\makesavenoteenv{longtable}
\usepackage{graphicx}
\makeatletter
\def\maxwidth{\ifdim\Gin@nat@width>\linewidth\linewidth\else\Gin@nat@width\fi}
\def\maxheight{\ifdim\Gin@nat@height>\textheight\textheight\else\Gin@nat@height\fi}
\makeatother
% Scale images if necessary, so that they will not overflow the page
% margins by default, and it is still possible to overwrite the defaults
% using explicit options in \includegraphics[width, height, ...]{}
\setkeys{Gin}{width=\maxwidth,height=\maxheight,keepaspectratio}
% Set default figure placement to htbp
\makeatletter
\def\fps@figure{htbp}
\makeatother

\makeatletter
\@ifpackageloaded{tcolorbox}{}{\usepackage[skins,breakable]{tcolorbox}}
\@ifpackageloaded{fontawesome5}{}{\usepackage{fontawesome5}}
\definecolor{quarto-callout-color}{HTML}{909090}
\definecolor{quarto-callout-note-color}{HTML}{0758E5}
\definecolor{quarto-callout-important-color}{HTML}{CC1914}
\definecolor{quarto-callout-warning-color}{HTML}{EB9113}
\definecolor{quarto-callout-tip-color}{HTML}{00A047}
\definecolor{quarto-callout-caution-color}{HTML}{FC5300}
\definecolor{quarto-callout-color-frame}{HTML}{acacac}
\definecolor{quarto-callout-note-color-frame}{HTML}{4582ec}
\definecolor{quarto-callout-important-color-frame}{HTML}{d9534f}
\definecolor{quarto-callout-warning-color-frame}{HTML}{f0ad4e}
\definecolor{quarto-callout-tip-color-frame}{HTML}{02b875}
\definecolor{quarto-callout-caution-color-frame}{HTML}{fd7e14}
\makeatother
\makeatletter
\@ifpackageloaded{bookmark}{}{\usepackage{bookmark}}
\makeatother
\makeatletter
\@ifpackageloaded{caption}{}{\usepackage{caption}}
\AtBeginDocument{%
\ifdefined\contentsname
  \renewcommand*\contentsname{Table of contents}
\else
  \newcommand\contentsname{Table of contents}
\fi
\ifdefined\listfigurename
  \renewcommand*\listfigurename{List of Figures}
\else
  \newcommand\listfigurename{List of Figures}
\fi
\ifdefined\listtablename
  \renewcommand*\listtablename{List of Tables}
\else
  \newcommand\listtablename{List of Tables}
\fi
\ifdefined\figurename
  \renewcommand*\figurename{Figure}
\else
  \newcommand\figurename{Figure}
\fi
\ifdefined\tablename
  \renewcommand*\tablename{Table}
\else
  \newcommand\tablename{Table}
\fi
}
\@ifpackageloaded{float}{}{\usepackage{float}}
\floatstyle{ruled}
\@ifundefined{c@chapter}{\newfloat{codelisting}{h}{lop}}{\newfloat{codelisting}{h}{lop}[chapter]}
\floatname{codelisting}{Listing}
\newcommand*\listoflistings{\listof{codelisting}{List of Listings}}
\usepackage{amsthm}
\theoremstyle{plain}
\newtheorem{lemma}{Lemma}[chapter]
\theoremstyle{remark}
\AtBeginDocument{\renewcommand*{\proofname}{Proof}}
\newtheorem*{remark}{Remark}
\newtheorem*{solution}{Solution}
\newtheorem{refremark}{Remark}[chapter]
\newtheorem{refsolution}{Solution}[chapter]
\makeatother
\makeatletter
\makeatother
\makeatletter
\@ifpackageloaded{caption}{}{\usepackage{caption}}
\@ifpackageloaded{subcaption}{}{\usepackage{subcaption}}
\makeatother

\ifLuaTeX
  \usepackage{selnolig}  % disable illegal ligatures
\fi
\usepackage[]{natbib}
\bibliographystyle{plainnat}
\usepackage{bookmark}

\IfFileExists{xurl.sty}{\usepackage{xurl}}{} % add URL line breaks if available
\urlstyle{same} % disable monospaced font for URLs
\hypersetup{
  pdftitle={Model Selection for Factor Analysis},
  pdfauthor={Zhining Wang},
  hidelinks,
  pdfcreator={LaTeX via pandoc}}

\usepackage{fancyhdr}
\usepackage{graphicx}
\usepackage{multirow}
\usepackage{amsmath}	% Advanced maths commands
\usepackage{amssymb}	% Extra maths symbols
\usepackage{rotating} % Rotating table
\usepackage[toc,page]{appendix}
\usepackage{setspace}
\usepackage{rotating}
\usepackage{longtable}
\usepackage{natbib}
\usepackage{multirow}


\def\TODO#1{{\color{red}{\bf [TODO:} {\it{#1}}{\bf ]}}}

%%%%%% AUTHORS - PLACE YOUR OWN PACKAGES HERE %%%%%

\usepackage{color}
\usepackage[normalem]{ulem}

% These are autoref macros
\def\chapterautorefname{Chapter}
\def\sectionautorefname{Section}
\def\subsectionautorefname{Section}
\def\subsubsectionautorefname{Section}
\def\figureautorefname{Figure}
\def\tableautorefname{Table}
\def\equationautorefname{equation}

% There are editing macros
\newcommand{\red}[1]{{\textcolor{red}{#1}}}
\newcommand{\blue}[1]{{\textcolor{blue}{#1}}}
\newcommand{\cyan}[1]{{\textcolor{cyan}{#1}}}
\newcommand{\cut}[1]{{\red{\sout{#1}}}}
\newcommand{\add}[1]{{\cyan{#1}}}

% Autoref macros
\def\sectionautorefname{Section}
\def\subsectionautorefname{Section}
\def\subsubsectionautorefname{Section}
\def\figureautorefname{Figure}
\def\tableautorefname{Table}
\def\equationautorefname{equation}
\newcommand{\aref}[1]{\hyperref[#1]{Appendix~\ref{#1}}}

% Only include extra packages if you really need them. Common packages are:
%\usepackage{caption}
%\usepackage{subcaption}
%\captionsetup{compatibility=false} %https://tex.stackexchange.com/questions/31906/subcaption-package-compatibility-issue
\usepackage{multicol}        % Multi-column entries in tables
\usepackage{booktabs} %to include pandas latex tables
\usepackage{float}



\title{Model Selection for Factor Analysis}
\author{Zhining Wang}
\date{28th August 2024}

\begin{document}
  \begin{frontmatter}
  \begin{titlepage}
  %%% TITLE PAGE:
  \pagenumbering{roman}  % first use Roman numerals for page numbers

  \begin{titlepage}
    \begin{flushright}%
      \vspace{50mm}
      {\small A thesis submitted for the degree of {\it Master of
  Mathematical Science}}
      \rule[1ex]{\textwidth}{1pt}\\
      {\fontsize{9}{0} 28th August 2024}\\
      \vspace{25mm}
      {\fontsize{40}{44}\bfseries Model Selection for Factor
  Analysis\par}
        \vspace{12mm}
        \vfill
      {\fontsize{20}{0}\bfseries Zhining Wang}\\
      \vspace{2mm}
      {\fontsize{8}{0} Mathematical Sciences Institute}\\
      \vspace{35mm}
      {\fontsize{10}{0}\bfseries supervised by}\\
      Dr.~Emi Tanaka, Dr.~Qinian Jin
      
      \vspace{2.0cm}
  		\includegraphics[width=0.4\textwidth]{\_extensions/anu-thesis/assets/latex/ANU\_Primary\_Horizontal\_GoldBlack.eps}\\
   \end{flushright}%

   \clearpage\thispagestyle{empty}
   \normalfont
   \vspace*{\fill}
   \noindent
   \begin{tabular}{lp{10cm}}
     {\bf Master of Mathematical Science Thesis} & \\[2mm]
     {\bf Author:} & Zhining Wang\\[2mm]
     {\bf Supervisors:} & Dr.~Emi Tanaka, Dr.~Qinian Jin\\[2mm]
     {\bf URL:} & true\\[2mm]
     {\bf Project period:} & 2024-03-01 -- 2024-10-01 \\[2mm]
   \end{tabular}\\[2mm]

  \noindent Mathematical Sciences Institute\\
  \noindent Australian National University

  \end{titlepage}
  \setlength{\parindent}{0pt}
  \setlength{\parskip}{1ex plus 0.5ex minus 0.2ex}








  \end{titlepage}
  \end{frontmatter}


\renewcommand*\contentsname{Table of contents}
{
\setcounter{tocdepth}{2}
\tableofcontents
}
\phantomsection\addcontentsline{toc}{section}{List of Figures}
\listoffigures
\phantomsection\addcontentsline{toc}{section}{List of Tables}
\listoftables

\setstretch{1.2}
\mainmatter
\bookmarksetup{startatroot}

\chapter*{Preface}\label{preface}
\addcontentsline{toc}{chapter}{Preface}

\markboth{Preface}{Preface}

\bookmarksetup{startatroot}

\chapter{Introduction}\label{introduction}

Multivariate data, where each observational unit consists of multiple
outcomes, are abundant across multitude of disciplines (e.g.~in
psychology, multiple test scores from different courses for a student;
in agriculture, yield measurements at different environments for the
same crop variety; and in biology, gene expressions from genes of the
same organism). A key analysis of multivariate data include
understanding or predicting the characteristics of observational units.
A simple analysis may assess one outcome at a time using some univariate
technique, however, a more sophisticated form of an analysis would
consider multiple outcomes simultaneously and exploit the correlated
structure within the data. While the latter analysis would be more
desirable to make the most out of the data, there are several challenges
as explained next.

Suppose we have a multivariate data denoted as a \(n \times p\) matrix,
\(\textbf{Y}\), where \(n \in \mathbb{Z}^+\) is the number of
observational units and \(p \in \mathbb{Z}^+\) is the number of
dependent variables. The entry in the \(i\)-th row and \(j\)-th column
of \(\textbf{Y}\) is written as \(y_{ij}\). We may write the matrix as
\(\mathbf{Y} = (y_{ij})\),
\(\mathbf{Y} = \begin{bmatrix}\boldsymbol{y}_{(1)} & \cdots & \boldsymbol{y}_{(p)}\end{bmatrix}\).

Notation used throughout this thesis is presented in
Table~\ref{tbl-notation}.

\begin{longtable}[]{@{}
  >{\raggedleft\arraybackslash}p{(\columnwidth - 2\tabcolsep) * \real{0.1600}}
  >{\raggedright\arraybackslash}p{(\columnwidth - 2\tabcolsep) * \real{0.8400}}@{}}
\caption{List of notations used in this
report.}\label{tbl-notation}\tabularnewline
\toprule\noalign{}
\begin{minipage}[b]{\linewidth}\raggedleft
Notation
\end{minipage} & \begin{minipage}[b]{\linewidth}\raggedright
Description
\end{minipage} \\
\midrule\noalign{}
\endfirsthead
\toprule\noalign{}
\begin{minipage}[b]{\linewidth}\raggedleft
Notation
\end{minipage} & \begin{minipage}[b]{\linewidth}\raggedright
Description
\end{minipage} \\
\midrule\noalign{}
\endhead
\bottomrule\noalign{}
\endlastfoot
\(\boldsymbol{A}^\top\) & the transpose of the matrix (or vector)
\(A\) \\
\(\mathbb{R}^{p}\) & the space of all \(p\)-dimensional real column
vectors like \([a_1,a_2,\dots,a_p]^\top\) \\
\(\mathbb{R}^{p\times q}\) & the space of all real matrices with size
\(p\times q\) \\
\(\mathbb{E}(\cdot)\) & the expectation, or mean of a random variable \\
\(\mathbb{V}(\cdot)\) & the variance, or covariance of a random
variable \\
\(\boldsymbol{Cov}(\cdot , \cdot)\) & the covariance of two random
variables \\
\(\boldsymbol{0}_{p}\) and \(\boldsymbol{0}_{p\times q}\) & the
\(p\)-dimensional \(0\) vector or \(0\) matrix in size \(p\times q\)
respectively \\
\(\boldsymbol{I}_p\) & the identity matrix in size \(p\times p\) \\
\(\det(\cdot)\) & the determinant of a matrix \\
\(\boldsymbol{m}_i\) & the \(i^{th}\) row of the matrix
\(\textbf{M}\) \\
\(\boldsymbol{m}_{(j)}\) & the \(j^{th}\) column of the matrix
\(\textbf{M}\) \\
\end{longtable}

This thesis is structured as follows. Chapter~\ref{sec-bg} outlines the
background information on factor analytics models, EM algorithm, and
penalised likelihood estimation. Chapter~\ref{sec-method} describe the
methods and simulation settings. Chapter~\ref{sec-results} presents the
results from the simulation with discussion.

This thesis is written reproducibly using the Quarto system
\citep{quarto} and all the code to reproduce the thesis and results can
be found at \url{https://github.com/ZhiningW/Master_Project}.

\bookmarksetup{startatroot}

\chapter{Background}\label{sec-bg}

\section{Factor analytic model}\label{factor-analytic-model}

Factor analysis (FA) is a statistical method which attempts to use fewer
underlying factors to explain the correlation between a large set of
observed variables \citep{mardiaMultivariateAnalysis1979}. FA provides a
useful tool for exploring the covariance structure among observable
variables \citep{hiroseSparseEstimationNonconcave2015}.

An FA model assumes that the observations are explained by a small
number of underlying factors. This assumption is suited to many areas,
e.g.~in psychology where certain variables, like intelligence, cannot be
directly measured \citep{mardiaMultivariateAnalysis1979}.

Suppose we have an observable random vector
\(\boldsymbol{y}_i\in \mathbb{R}^p\) for the \(i\)-th subject with mean
\(\mathbb{E}(\boldsymbol{y}_i)=\boldsymbol{\mu}\) and variance
\(\mathbb{V}(\boldsymbol{y}_i)=\boldsymbol{\Sigma}\). Then a \(k\)-order
factor analysis model for \(\boldsymbol{y}_i\) can be given by \[
\begin{equation}
\boldsymbol{y}_i=\boldsymbol{\mu}+\boldsymbol{\Lambda} \boldsymbol{f}_i+\boldsymbol{\epsilon}_i,
\end{equation} 
\]

where \(\boldsymbol{\Lambda} \in \mathbb{R}^{p \times k}\),
\(\boldsymbol{f}_i \in \mathbb{R}^{k}\) is and
\(\boldsymbol{\epsilon}_i \in \mathbb{R}^{p}\) are called the
\emph{loading matrix}, \emph{common factors} and \emph{unique (or
specific) factors}, respectively. The order \(k\) is usually much
smaller than \(p\). For simplicity, we conduct a centralization to those
\(\boldsymbol{y}_i\) and assume that
\(\boldsymbol{\mu} = \boldsymbol{0}_p\).

To make the model well-defined, we may assume

\[\mathbb{E}\left(\begin{bmatrix}\boldsymbol{f}_i\\\boldsymbol{\epsilon}_i\end{bmatrix}\right) = \begin{bmatrix}\boldsymbol{0}_k\\\boldsymbol{0}_p\end{bmatrix}\quad\text{and}\quad\mathbb{V}\left(\begin{bmatrix}\boldsymbol{f}_i\\\boldsymbol{\epsilon}_i\end{bmatrix}\right) =\begin{bmatrix}\mathbf{I}_k & \mathbf{0}_{k\times p}\\\mathbf{0}_{p\times k} & \mathbf{\Psi}\end{bmatrix},\]

where \(\boldsymbol{\Psi}\) is a \(p\times p\) diagonal matrix where we
denote the \(i\)-th diagonal entry as \(\psi_{ii}\). Based on this
assumption, the covariance of observable vector \(\boldsymbol{y}_i\) can
be modelled by

\[\mathbb{V}(\boldsymbol{y}_i|\boldsymbol{\Lambda},\boldsymbol{\Psi} )=\boldsymbol{\Lambda}\boldsymbol{\Lambda}^\top+\boldsymbol{\Psi}.\]

Alternatively, we can write the model as

\[\boldsymbol{y} = (\mathbf{I}_n \otimes \boldsymbol{\Lambda})\boldsymbol{f} + \boldsymbol{\epsilon}\]
where \(\boldsymbol{y} =
\begin{pmatrix}
\boldsymbol{y}_1 \\ \boldsymbol{y}_2 \\ \vdots \\ \boldsymbol{y}_n
\end{pmatrix}\), \(\boldsymbol{f} =
\begin{pmatrix}
\boldsymbol{f}_1 \\ \boldsymbol{f}_2 \\ \vdots \\  \boldsymbol{f}_n
\end{pmatrix}\) and \(\boldsymbol{\epsilon} =
\begin{pmatrix}
\boldsymbol{\epsilon}_1 \\ \boldsymbol{\epsilon}_2 \\ \vdots \\  \boldsymbol{\epsilon}_n
\end{pmatrix}\). Thus we have

\[\mathbb{E}\left(\begin{bmatrix}\boldsymbol{f}\\\boldsymbol{\epsilon}\end{bmatrix}\right) = \begin{bmatrix}\boldsymbol{0}_{nk}\\\boldsymbol{0}_{np}\end{bmatrix}\quad\text{and}\quad\mathbb{V}\left(\begin{bmatrix}\boldsymbol{f}\\\boldsymbol{\epsilon}\end{bmatrix}\right) =\begin{bmatrix}\mathbf{I}_{nk} & \mathbf{0}_{nk\times np}\\\mathbf{0}_{np\times nk} & \mathbf{I}_n \otimes \mathbf{\Psi}\end{bmatrix}.\]

\section{Indeterminacy of the loading
matrix}\label{indeterminacy-of-the-loading-matrix}

One can easily see that if our factor analytic model is given by (1),
then it can also be modelled as
\[\boldsymbol{y}=(\boldsymbol{\Lambda}\boldsymbol{M})(\boldsymbol{M}^\top\boldsymbol{f}) +\boldsymbol{\mu}+\boldsymbol{\epsilon}\]
where the matrix \(\boldsymbol{M}\) is orthogonal and simultaneously the
variance of \(\boldsymbol{y}\) given by (2) still holds, since
\[\mathbb{V}[\boldsymbol{y}]=(\boldsymbol{\Lambda}\boldsymbol{M}\boldsymbol{M}^\top)\mathbb{V}[\boldsymbol{f}](\boldsymbol{\Lambda}\boldsymbol{M}\boldsymbol{M}^\top)^\top+\boldsymbol{\Psi}=\boldsymbol{\Lambda}\boldsymbol{\Lambda}^\top+\boldsymbol{\Psi}.\]
Therefore a rotated loading matrix
\(\boldsymbol{\Lambda}\boldsymbol{M}\) is still a valid loading matrix
for a factor analytic model. Sometimes we resolve this problem by making
the loading matrix to satisfy some constraints like
\citep{mardiaMultivariateAnalysis1979}
\[\boldsymbol{\Lambda}^\top \boldsymbol{\Psi}^{-1} \boldsymbol{\Lambda} \text{ is diagonal.}\]

\section{Parameter Estimation}\label{parameter-estimation}

We denote the set of parameters by
\(\boldsymbol{\theta} = \{\text{vec}(\boldsymbol{\Lambda}), \text{diag}(\boldsymbol{\Psi})\}\)
where \(\text{vec}(\cdot)\) is the vectorisation of the input matrix and
\(\text{diag}(\cdot)\) is the diagonal elements of the input matrix.

Traditionally, a two-step procedure is used to construct a factor
analytic model: estimate parameters by maximum likelihood estimation
(aka, MLE) and then use rotation techniques to find an interpretable
model.

\subsection{Maximum Likelihood
Estimation}\label{maximum-likelihood-estimation}

Suppose we have \(n\) independent and identically distributed
observations
\(\boldsymbol{y}_1,\boldsymbol{y}_2,\dots,\boldsymbol{y}_n\) from a
\(p\)-dimensional multi-variate normal distribution
\(\mathcal{N}_p(\boldsymbol{\mu},\boldsymbol{\Sigma})\). Now denote
\(f_{\boldsymbol{y}}(\boldsymbol{y};\boldsymbol{\theta})\) as the
probability density function of the random vector \(\boldsymbol{Y}\)
corresponding to \(\boldsymbol{y}\). Then the likelihood is given by

\begin{equation}\phantomsection\label{eq-likelihood}{
L(\boldsymbol{\theta};\boldsymbol{y})=f_{\boldsymbol{y}}(\boldsymbol{y};\boldsymbol{\theta}) = \prod^n_{i=1}f_{\boldsymbol{y}_i}(\boldsymbol{y}_i;\boldsymbol{\theta}) =  \prod^n_{i=1}\left[(2\pi)^{-\frac{p}{2}}\det(\boldsymbol{\Sigma})^{-\frac{1}{2}}\exp(-\frac{1}{2}(\boldsymbol{y}_i-\boldsymbol{\mu})^\top\boldsymbol{\Sigma}^{-1}(\boldsymbol{y}_i-\boldsymbol{\mu}))\right].
}\end{equation}

The maximum likelihood estimate (MLE) of \(\boldsymbol{\theta}\),
denoted as \(\hat{\boldsymbol{\theta}}\), is found by finding
\(\boldsymbol{\theta}\) that maximizes Equation~\ref{eq-likelihood}.
However, it is often more convenient to maximize the log likelihood
function. To be more computational friendly, a better form of log
likelihood is given by

\begin{lemma}[]\protect\hypertarget{lem-MLE}{}\label{lem-MLE}

The log-likelihood is given by

\begin{equation}\phantomsection\label{eq-loglikelihood}{\ell(\boldsymbol{\theta}) =  -\frac{n}{2}\left[p\log(2\pi)+\log(\det(\boldsymbol{\Sigma}))+\text{tr}(\boldsymbol{\Sigma}^{-1}\boldsymbol{S})\right]}\end{equation}

where \(\boldsymbol{S}\) is the sample covariance defined as
\(\boldsymbol{S}=\frac{1}{n}\sum^n_{i=1}(\boldsymbol{y}_i-\boldsymbol{\mu})(\boldsymbol{y}_i-\boldsymbol{\mu})^\top\).

\end{lemma}

\begin{proof}
\[
\begin{align*}
\ell(\boldsymbol{\theta})=& \sum^n_{i=1}\left[-\frac{p}{2}\log(2\pi)-\frac{1}{2}\log(\det(\boldsymbol{\Sigma}))-\frac{1}{2}(\boldsymbol{y}_i-\boldsymbol{\mu})^\top\boldsymbol{\Sigma}^{-1}(\boldsymbol{y}_i-\boldsymbol{\mu})\right]\\
=& -\frac{n}{2}\left[p\log(2\pi)+\log(\det(\boldsymbol{\Sigma}))+\frac{1}{n}\sum_{i=1}^n(\boldsymbol{y}_i-\boldsymbol{\mu})^\top\boldsymbol{\Sigma}^{-1}(\boldsymbol{y}_i-\boldsymbol{\mu})\right]\\
=& -\frac{n}{2}\left[p\log(2\pi)+\log(\det(\boldsymbol{\Sigma}))+\text{tr}(\boldsymbol{\Sigma}^{-1}\boldsymbol{S})\right].
\end{align*}
\] The last equality holds from the following fact: \[
\begin{align*}
&\boldsymbol{S}=\frac{1}{n}\sum^n_{i=1}(\boldsymbol{y}_i-\boldsymbol{\mu})(\boldsymbol{y}_i-\boldsymbol{\mu})^\top\\
\Leftrightarrow \ & \boldsymbol{\Sigma}^{-1}\boldsymbol{S}=\frac{1}{n}\sum^n_{i=1}\boldsymbol{\Sigma}^{-1}(\boldsymbol{y}_i-\boldsymbol{\mu})(\boldsymbol{y}_i-\boldsymbol{\mu})^\top\\
\Leftrightarrow \ & \text{tr}(\boldsymbol{\Sigma}^{-1}\boldsymbol{S}) = \frac{1}{n} \sum^n_{i=1} \text{tr} (\boldsymbol{\Sigma}^{-1}(\boldsymbol{y}_i-\boldsymbol{\mu})(\boldsymbol{y}_i-\boldsymbol{\mu})^\top)\\
\Leftrightarrow \ & \text{tr}(\boldsymbol{\Sigma}^{-1}\boldsymbol{S}) = \frac{1}{n}\sum_{i=1}^n(\boldsymbol{y}_i-\boldsymbol{\mu})^\top\boldsymbol{\Sigma}^{-1}(\boldsymbol{y}_i-\boldsymbol{\mu})
\end{align*}
\]
\end{proof}

The MLE is obtained by seeking the roots of equation system (provided
the likelihood is compact)

\begin{equation}\phantomsection\label{eq-solveMLE}{
\begin{align*}
\frac{\partial}{\partial \boldsymbol{\Lambda}}\ell(\boldsymbol{\theta})&=0 \\
\frac{\partial}{\partial \boldsymbol{\Psi}}\ell(\boldsymbol{\theta})&= 0.
\end{align*}
}\end{equation}

However, there is no closed form of the roots. Many iterative algorithms
are developed to get the roots of the equation, like
Expectation-Maximization (EM) algorithm \citep{Rubin1982EMAlgorithms},
we will discuss this algorithm in the last part of background.

\section{Rotation Techniques}\label{rotation-techniques}

After estimation, we want to rotate the loading matrix to possess a
sparse matrix in order to interpret the observable variables by
underlying factors better. Also there are many method to achieve
rotation as well such as the varimax method and the promax method
\citep{hiroseSparseEstimationNonconcave2015}.

Suppose \(Q(\boldsymbol{\Lambda})\) is an criterion for
\(\boldsymbol{\Lambda}\) in the rotation procedure, and we may express
it as
\(Q(\boldsymbol{\Lambda}):= \sum^p_{i=1}\sum^d_{j=1}P(\boldsymbol{\Lambda}_{ij})\)
where \(P(\cdot)\) is some loss
function\citep{hiroseSparseEstimationNonconcave2015}. Specifically, if
we set \(P(\cdot)=|\cdot|\), we have LASSO to generate a theoretically
sparse loading matrix. If we rewrite this in a optimization problem, it
can be given as \citep{Jennrich2004Rotation}

\[
\begin{align*}
&\min_{\boldsymbol{\Lambda}} \sum^p_{i=1}\sum^k_{j=1}P(\boldsymbol{\Lambda}_{ij})\\
&\text{subject to } \boldsymbol{\Lambda}=\boldsymbol{\Lambda}_0\boldsymbol{M} \text{ and } \boldsymbol{M}^\top\boldsymbol{M}=\boldsymbol{I}_k,
\end{align*}
\]

where \(\boldsymbol{\Lambda}_0\) is an initial guess. Since we execute
this technique after obtaining the MLE of parameters, therefore what we
want is

\[
\begin{align*}
&\min_{\boldsymbol{\Lambda}} \sum^p_{i=1}\sum^k_{j=1}P(\boldsymbol{\Lambda}_{ij})\\
&\text{subject to } \boldsymbol{\Lambda}=\hat{\boldsymbol{\Lambda}}_{\text{MLE}}\boldsymbol{M} \text{ and } \boldsymbol{M}^\top\boldsymbol{M}=\boldsymbol{I}_k,
\end{align*}
\]

where \(\hat{\boldsymbol{\Lambda}}_{\text{MLE}}\) is the maximum
likelihood estimator.

\subsection{Discussion about two-step
method}\label{discussion-about-two-step-method}

The traditional two-step method faces significant shortcomings,
primarily its unsuitability
\citep{hiroseSparseEstimationNonconcave2015}. Similar to challenges in
regression models, MLE can result in overfitting, and rotation
techniques might not yield a sufficiently sparse loading matrix. See
this as an example: Suppose the true loading matrix is given by \[
\boldsymbol{\Lambda}=
\begin{bmatrix}
0.8 & 0 \\
0 & 0.8 \\
0.6 & 0\\
0 & 0.7\\
0 & 0.6\\
\end{bmatrix}.
\] and the real \(\boldsymbol{\Psi}\) is given by \[
\boldsymbol{\Psi}=\text{diag}(0.1,0.2,0.2,0.1,0.1).
\] We generate \(\boldsymbol{Y}\in \mathbb{R}^{{10000}\times 5}\) from a
multivariate normal distribution with mean \(\mathbb{0}_{5}\) and
covariance
\(\boldsymbol{\Lambda}\boldsymbol{\Lambda}^\top+\boldsymbol{\Psi}\). We
utilize the R-function \texttt{factanal()} to produce the loading matrix
via MLE, where this R-function will unitize varimax method to achieve
rotation automatically. What we finally get is

\[
\hat{\boldsymbol{\Lambda}}=
\begin{bmatrix}
0 & 0.997\\
0.872 & 0\\
0 & 0.748\\
0.915 & 0\\
0.882 & 0\\
\end{bmatrix},
\hat{\boldsymbol{\Psi}}=\text{diag}(0.005,0.240,0.440,0.163,0.223)
\] This result gives the same sparsity result. However, the result of
loading matrix in the first and last two dimensions are much larger than
the real one, which implies overfitness.

Simulation code:

\begin{Shaded}
\begin{Highlighting}[]
\FunctionTok{set.seed}\NormalTok{(}\DecValTok{123}\NormalTok{)}
\NormalTok{n }\OtherTok{\textless{}{-}} \DecValTok{10000} \CommentTok{\# Number of observations}
\CommentTok{\# True loading matrix}
\NormalTok{true\_loading }\OtherTok{\textless{}{-}} \FunctionTok{matrix}\NormalTok{(}\FunctionTok{c}\NormalTok{(}\FloatTok{0.8}\NormalTok{,}\DecValTok{0}\NormalTok{,}\FloatTok{0.6}\NormalTok{,}\DecValTok{0}\NormalTok{,}\DecValTok{0}\NormalTok{,}\DecValTok{0}\NormalTok{,}\FloatTok{0.8}\NormalTok{,}\DecValTok{0}\NormalTok{,}\FloatTok{0.7}\NormalTok{,}\FloatTok{0.6}\NormalTok{),}\AttributeTok{nrow=}\DecValTok{5}\NormalTok{,}\AttributeTok{ncol=}\DecValTok{2}\NormalTok{) }
\CommentTok{\# True psi matrix}
\NormalTok{true\_psi }\OtherTok{\textless{}{-}} \FunctionTok{diag}\NormalTok{(}\FunctionTok{c}\NormalTok{(}\FloatTok{0.1}\NormalTok{,}\FloatTok{0.2}\NormalTok{,}\FloatTok{0.2}\NormalTok{,}\FloatTok{0.1}\NormalTok{,}\FloatTok{0.1}\NormalTok{))}
\CommentTok{\# Generate samples from multivariate normal distribution}
\NormalTok{samples }\OtherTok{\textless{}{-}}\NormalTok{ MASS}\SpecialCharTok{::}\FunctionTok{mvrnorm}\NormalTok{(}\AttributeTok{n =}\NormalTok{ n, }\AttributeTok{mu =} \FunctionTok{rep}\NormalTok{(}\DecValTok{0}\NormalTok{, }\DecValTok{5}\NormalTok{), }
                         \AttributeTok{Sigma =}\NormalTok{ true\_loading }\SpecialCharTok{\%*\%} \FunctionTok{t}\NormalTok{(true\_loading) }\SpecialCharTok{+}\NormalTok{ true\_psi)}
\CommentTok{\# Traditional two step procedure}
\CommentTok{\# the function factanal has already used varimax method to get the optimal loading}
\NormalTok{fa\_result }\OtherTok{\textless{}{-}} \FunctionTok{factanal}\NormalTok{(}\AttributeTok{factors =} \DecValTok{2}\NormalTok{, }\AttributeTok{covmat =} \FunctionTok{cor}\NormalTok{(samples))}
\CommentTok{\# Result}
\NormalTok{fa\_result}
\end{Highlighting}
\end{Shaded}

\begin{verbatim}

Call:
factanal(factors = 2, covmat = cor(samples))

Uniquenesses:
[1] 0.005 0.240 0.440 0.163 0.222

Loadings:
     Factor1 Factor2
[1,]          0.997 
[2,]  0.872         
[3,]          0.748 
[4,]  0.915         
[5,]  0.882         

               Factor1 Factor2
SS loadings      2.375   1.555
Proportion Var   0.475   0.311
Cumulative Var   0.475   0.786

The degrees of freedom for the model is 1 and the fit was 2e-04 
\end{verbatim}

Another problem is that the absence of model selection complicates
efficient modeling by making it difficult to determine the precise
number of underlying factors, that is, the order of the factor analytic
model. A usual but troublesome method is conducting a hypothesis test
for the model \citep{mardiaMultivariateAnalysis1979}. We have
\begin{align*}
  &&&H_k: k \text{ common factors are sufficient to describe the data } \\
  \longleftrightarrow &&&H_a: \text{ Otherwise}.
\end{align*} The test statistics is given by
\(\text{TS}=nF(\hat{\boldsymbol{\Lambda}},\hat{\boldsymbol{\Psi}})\)
which has an asymptotic \(\chi^2_s\) distribution with
\(s=\frac{1}{2}(p-k)^2-(p+k)\) by the property of MLE, where \(F\) is
given by (\citet{mardiaMultivariateAnalysis1979}) \[
F(\boldsymbol{\Lambda},\boldsymbol{\Psi})=\text{tr}(\boldsymbol{\Sigma}^{-1}\boldsymbol{S})-\log(\det(\boldsymbol{\Sigma}^{-1}\boldsymbol{S}))-p.
\] Typically, the test is conducted at a \(5\%\) significant level. We
can start the procedure from a very small \(k\), say, \(k=1\) or \(k=2\)
and then increase \(k\) until the null hypothesis is not rejected.

\section{Penalized Likelihood Method}\label{penalized-likelihood-method}

Penalized likelihood method can be viewed as a generalization of
two-step method mentioned above
\citep{hiroseSparseEstimationNonconcave2015}. A penalized factor
analytic model can be obtained by solving following optimization problem
\[
 \max_{\boldsymbol{\Lambda},\boldsymbol{\Psi}}\ell_p := \ell(\boldsymbol{\theta})-P(\boldsymbol{\theta};\rho).
\] where we call \(\ell_p\) as the penalized likelihood, \(\rho\) is
called regularization parameter or tuning parameter, and we treat
\(P(\boldsymbol{\theta};\rho)\) as the penalty. There are many types of
penalized functions developed, such as LASSO
(\(P(\boldsymbol{\Lambda})=\sum_{i=1}^p\sum_{j=1}^k|\lambda_{ij}|\)) and
MC+
(\$P(\boldsymbol{\theta};\rho;\gamma)=\sum\emph{\{i=1\}\^{}p\sum}\{j=1\}\^{}k
\{
n(\textbar{}\lambda\emph{\{ij\}\textbar-\frac{\lambda_{ij}^2}{2\rho\gamma})I(\textbar{}\lambda}\{ij\}\textbar\textless{}\rho\gamma)+\frac{\rho^2\gamma}{2}I(\textbar{}\lambda\_\{ij\}\textbar{}\geq\rho\gamma)\}
\$)\citep{hiroseSparseEstimationNonconcave2015},where \(\gamma\) is
another tuning parameter to keep a balance between unbiasedness and
concavity. In this article, we will mainly focus on the LASSO penalty.

\subsection{LASSO penalty and LASSO
estimator}\label{lasso-penalty-and-lasso-estimator}

Again, recall that the LASSO penalized likelihood is given by
\[\ell_p=-\frac{n}{2}[p\log(2\pi)+\log(\det(\boldsymbol{\Sigma}))+\text{tr}(\boldsymbol{\Sigma}^{-1}\boldsymbol{S})]-\rho\sum^p_{i=1}\sum^k_{j=1}|\lambda_{ij}|\]
and the LASSO estimator, denoted as
\((\boldsymbol{\Lambda}^*,\boldsymbol{\Psi}^*)\), can be obtained via

\[
\begin{align*}
(\boldsymbol{\Lambda}^*,\boldsymbol{\Psi}^*) &= \text{arg}\max_{\boldsymbol{\Lambda},\boldsymbol{\Psi}} \quad -\frac{n}{2}[p\log(2\pi)+\log(\det(\boldsymbol{\Sigma}))+\text{tr}(\boldsymbol{\Sigma}^{-1}\boldsymbol{S})]-\rho\sum^p_{i=1}\sum^k_{j=1}|\lambda_{ij}|\\
&=\text{arg}\min_{\boldsymbol{\Lambda},\boldsymbol{\Psi}} \quad \log(\det(\boldsymbol{\Sigma}))+\text{tr}(\boldsymbol{\Sigma}^{-1}\boldsymbol{S})+\frac{2}{n}\rho\sum^p_{i=1}\sum^k_{j=1}|\lambda_{ij}|.
\end{align*}
\] Nevertheless, the objective function is non\_differentiable at some
point because of the penalization term. Therefore we need to find other
approaches to obtain the estimator instead of solving

\section{The EM Algorithm}\label{sec-EM}

The Expectation-Maximization (EM) algorithm is a widely used iterative
method to find the MLE, especially when the model depends on latent
variables. The EM algorithm iteratively apply two distinct steps: the
Expectation step (E-step) and the Maximization step (M-step). More
specifically, we define

\[Q(\boldsymbol{\theta}|\boldsymbol{\theta}^{(t)}) = \mathbb{E}\left(\ell (\boldsymbol{\theta}; \boldsymbol{y})\bigg|\boldsymbol{y},\boldsymbol{\theta}^{(t)}\right).\]
Then for the \((t + 1)\)-th interation, the steps involve:

\begin{itemize}
\tightlist
\item
  \emph{E-step}: Compute
  \(Q(\boldsymbol{\theta}|\boldsymbol{\theta}^{(t)})\).
\item
  \emph{M-step}: Update \(\boldsymbol{\theta}^{(t+1)}\) as
  \(\boldsymbol{\theta}\) that maximses
  \(Q(\boldsymbol{\theta}|\boldsymbol{\theta}^{(t)})\).
\end{itemize}

The M-step may be replaced with updating \(\boldsymbol{\theta}^{(t+1)}\)
such that
\(Q(\boldsymbol{\theta}^{(t+1)}|\boldsymbol{\theta}^{(t)}) \geq Q(\boldsymbol{\theta}^{(t)}|\boldsymbol{\theta}^{(t)})\).

We iterate between E-step and M-step until convergence. The convergence
may be determined by a criteria such as
\(||\boldsymbol{\theta}^{(t+1)}-\boldsymbol{\theta}^{(t)}||_p\leq \epsilon\)
for some \(p\)-norm \(||\cdot||_p\) and \(\epsilon > 0\).

The EM algorithm approaches the problem of solving the likelihood
equation indirectly by proceeding iteratively in terms of
\(\ell(\boldsymbol{\theta};\boldsymbol{y})\). But it is unobservable
since it includes missing part of the data, then we use the conditional
expectation given \(\boldsymbol{y}\) and current fit for
\(\boldsymbol{\theta}\).

Now we need to show the correctness of the EM-algorithm. In other words,
by iteratively processing the EM-algorithm, the likelihood will keep
increasing (at least non-decreasing). First we need to show the
following lemma.

\begin{lemma}[Convergence of the EM
Algorithm]\protect\hypertarget{lem-convergence}{}\label{lem-convergence}

Suppose the likelihood is upper-bounded. Then for all \(\epsilon > 0\),
there exists \(t > t_0\) such that
\(||\boldsymbol{\theta}^{(t)} - \hat{\boldsymbol{\theta}}|| < \epsilon\)
where \(\hat{\boldsymbol{\theta}}\) is the MLE.

\end{lemma}

\begin{proof}
By the definition of conditional likelihood, our likelihood of complete
data, i.e.~the observable and unobservable data, can be expressed by \[
L_{\boldsymbol{X}}(\boldsymbol{\theta}) = f_{\boldsymbol{X}}(\boldsymbol{x};\boldsymbol{\theta})=L_{\boldsymbol{Y}}(\boldsymbol{\theta})f_{\boldsymbol{X|Y}}(\boldsymbol{x}|\boldsymbol{y};\boldsymbol{\theta}),
\] and hence the log-likelihood is given by \[
\log L_{\boldsymbol{X}}(\boldsymbol{\theta}) = \ell_{\boldsymbol{y}}(\boldsymbol{\theta}) + \log f_{\boldsymbol{X|Y}}(\boldsymbol{x}|\boldsymbol{y};\boldsymbol{\theta}).
\] Take expectation to both sides of the equation with respect to
\(\boldsymbol{x|y}\) and replace \(\boldsymbol{\theta}\) by
\(\boldsymbol{\theta}^{(k)}\), we will have \[
Q(\boldsymbol{\theta};\boldsymbol{\theta}^{(k)}) = \ell_{\boldsymbol{y}}(\boldsymbol{\theta}) + \mathbb{E}_{\boldsymbol{X}}[\log f_{\boldsymbol{X|Y}}(\boldsymbol{x}|\boldsymbol{y};\boldsymbol{\theta})|\boldsymbol{y},\boldsymbol{\theta}^{(k)}].
\] Now consider the difference of log-likelihood of \(\boldsymbol{Y}\)
function between two iterations, we have \[
\begin{align*}
  \ell_{\boldsymbol{y}}(\boldsymbol{\theta}^{(k+1)})-\ell_{\boldsymbol{y}}(\boldsymbol{\theta}^{(k)}) =
  &\{Q(\boldsymbol{\theta}^{(k+1)};\boldsymbol{\theta}^{(k)})-Q(\boldsymbol{\theta}^{(k)};\boldsymbol{\theta}^{(k)})\}\\
  &-\{\mathbb{E}_{\boldsymbol{X}}[\log         f_{\boldsymbol{X|Y}}(\boldsymbol{x}|\boldsymbol{y};\boldsymbol{\theta}^{(k+1)})|\boldsymbol{y},\boldsymbol{\theta}^{(k)}]\\
  &-\mathbb{E}_{\boldsymbol{X}}[\log         f_{\boldsymbol{X|Y}}(\boldsymbol{x}|\boldsymbol{y};\boldsymbol{\theta}^{(k)}|\boldsymbol{y},\boldsymbol{\theta}^{(k)}]\}.
\end{align*}
\] By the procedure of EM-algorithm, we always have
\(Q(\boldsymbol{\theta}^{(k+1)};\boldsymbol{\theta}^{(k)})\geq Q(\boldsymbol{\theta}^{(k)};\boldsymbol{\theta}^{(k)})\).
By the Gibbs's inequality, we have
\(\mathbb{E}_{\boldsymbol{X}}[\log         f_{\boldsymbol{X|Y}}(\boldsymbol{x}|\boldsymbol{y};\boldsymbol{\theta}^{(k+1)})|\boldsymbol{y},\boldsymbol{\theta}^{(k)}] \leq \mathbb{E}_{\boldsymbol{X}}[\log         f_{\boldsymbol{X|Y}}(\boldsymbol{x}|\boldsymbol{y};\boldsymbol{\theta}^{(k)}|\boldsymbol{y},\boldsymbol{\theta}^{(k)}].\)
Therefore during iterations, the sequnence of log-likelihood of observed
data \(\boldsymbol{Y}\) given by
\(\{\ell_{\boldsymbol{y}}(\boldsymbol{\theta}^{(n)})\}_{n\geq 1}:=\{\ell_{\boldsymbol{y}}(\boldsymbol{\theta}^{(1)}),\ell_{\boldsymbol{y}}(\boldsymbol{\theta}^{(2)}),\ell_{\boldsymbol{y}}(\boldsymbol{\theta}^{(3)}),\dots\}\),
is increasing. Now replace the term
\(\ell_{\boldsymbol{y}}(\boldsymbol{\theta}^{(k+1)})\) by
\(\ell_{\boldsymbol{y}}(\hat{\boldsymbol{{\theta}}})\), we will have \[
\ell_{\boldsymbol{y}}(\boldsymbol{\theta}^{(k)}) \leq \ell_{\boldsymbol{y}}(\hat{\boldsymbol{{\theta}}})
\] for all \(k=1,2,\dots\). Therefore the upper-bound of the sequence
\(\{\ell_{\boldsymbol{y}}(\boldsymbol{\theta}^{(n)})\}_{n\geq 1}\) is
\(\ell_{\boldsymbol{y}}(\hat{\boldsymbol{\theta}})\). By the monotone
convergence theorem, and the definition of convergence, we finished.
\end{proof}

\bookmarksetup{startatroot}

\chapter{Method}\label{sec-method}

As described in Chapter~\ref{sec-bg}, there are various methods to fit a
factor analytic model to multivariate data.

\section{Parameter estimation}\label{parameter-estimation-1}

\subsection{The EM Algorithm}\label{the-em-algorithm}

We treat the common factor matrix \(\boldsymbol{f}\) as the latent
variable. In E-step, we need to compute the conditional expectation of
the joint likelihood of \((\boldsymbol{y},\boldsymbol{f})\) given
\(\boldsymbol{y}\) and \(\boldsymbol{\theta}_{(t)}\), where
\(\boldsymbol{\theta}_{(t)}\) is the parameter we got in the \(t\)-th
iteration (\(t>1\)) and \(\boldsymbol{\theta}_{(0)}\) is the initial
value we set. In M-step, we maximize the conditional expectation over
parameters and get \(\boldsymbol{\theta}_{(t + 1)}\).

\subsubsection{E-Step}\label{e-step}

\begin{lemma}[]\protect\hypertarget{lem-fa-likelihood}{}\label{lem-fa-likelihood}

The joint log-likelihood of \((\boldsymbol{y},\boldsymbol{f})\) is given
by
\[\ell_{\boldsymbol{y},\boldsymbol{f}}(\boldsymbol{\theta})= -\frac{n}{2} \sum_{j=1}^p\log\psi_{jj}-\frac{1}{2}\sum_{i=1}^n\sum_{j=1}^p \frac{(y_{ij} - \sum_{k=1}^m \lambda_{jk}f_{ik})^2}{\psi_{jj}} - \sum_{i=1}^n\sum_{j=1}^pf_{ij}^2 + \text{constant}.\\\]

\end{lemma}

\begin{proof}
First, the joint log-likelihood of \((\boldsymbol{y},\boldsymbol{f})\)
is given by
\[\ell_{\boldsymbol{y},\boldsymbol{f}}(\boldsymbol{\theta})=\sum_{i=1}^n\log f(\boldsymbol{y}_i,\boldsymbol{f}_i) =\sum_{i=1}^n\log \left(f(\boldsymbol{y}_i|\boldsymbol{f}_i)f(\boldsymbol{f}_i)\right)\]
where
\(\boldsymbol{y}_i|\boldsymbol{f}_i \sim \mathcal{N}(\boldsymbol{\Lambda}\boldsymbol{f}_i, \boldsymbol{\Psi})\)
and
\(\boldsymbol{f}_i \sim \mathcal{N}(\boldsymbol{0}_m,\boldsymbol{I}_m)\).
Therefore we have

\$\$

\begin{align*}
\ell(\boldsymbol{\theta})=& \sum_{i=1}^n\log f(\boldsymbol{y}_i|\boldsymbol{f}_i) + \sum_{i=1}^n\log f(\boldsymbol{f}_i)\\
=& -\frac{np}{2}\log(2\pi)-\frac{n}{2}\log\left[\det(\boldsymbol{\Psi})\right]-\frac{1}{2}\sum_{i=1}^n(\boldsymbol{y}_i-\boldsymbol{\Lambda}\boldsymbol{f}_i)^\top\boldsymbol{\Psi}^{-1}(\boldsymbol{y}_i-\boldsymbol{\Lambda}\boldsymbol{f}_i)\\
&\quad -\frac{kp}{2}\log(2\pi)-\frac{1}{2}\log\left[\det(\boldsymbol{I}_m)\right]-\frac{1}{2}(\boldsymbol{f}_i-\boldsymbol{0}_k)^\top\boldsymbol{I}_m^{-1}(\boldsymbol{f}_i-\boldsymbol{0}_m)\\
=& -\frac{(n+k)p}{2}\log(2\pi)-\frac{n}{2}\sum_{j=1}^p\log \psi_{jj}-\frac{1}{2}\sum_{i=1}^n(\boldsymbol{y}_i-\boldsymbol{\Lambda}\boldsymbol{f}_i)^\top\boldsymbol{\Psi}^{-1}(\boldsymbol{y}_i-\boldsymbol{\Lambda}\boldsymbol{f}_i)- \frac{1}{2}\sum_{i=1}^n\boldsymbol{f}_i^\top\boldsymbol{f}_i\\
=& -\frac{(n+k)p}{2}\log(2\pi)-\frac{n}{2} \sum_{j=1}^p\log\psi_{jj}-\frac{1}{2}\sum_{i=1}^n\sum_{j}^p \frac{y_{ij}^2}{\psi_{jj}} + \frac{1}{2}\sum_{i=1}^n \boldsymbol{f}_i^\top\mathbf{\Lambda}^\top\mathbf{\Psi}^{-1}\boldsymbol{y}_i+ \frac{1}{2}\sum_{i=1}^n\boldsymbol{y}_i^\top\mathbf{\Psi}^{-1}\mathbf{\Lambda}\boldsymbol{f}_i  - \frac{1}{2}\sum_{i=1}^n
\boldsymbol{f}_i^\top\mathbf{\Lambda}^\top\mathbf{\Psi}^{-1}\mathbf{\Lambda}\boldsymbol{f}_i -\frac{1}{2}\sum_{i=1}^n\boldsymbol{f}_i^\top\boldsymbol{f}_i\\
=& -\frac{(n+k)p}{2}\log(2\pi)-\frac{n}{2} \sum_{j=1}^p\log\psi_{jj}-\frac{1}{2}\sum_{i=1}^n\sum_{j}^p \frac{y_{ij}^2}{\psi_{jj}} +   \sum_{i=1}^n \boldsymbol{y}_i^\top\mathbf{\Psi}^{-1}\mathbf{\Lambda}\boldsymbol{f}_i  - \frac{1}{2}\sum_{i=1}^n
\boldsymbol{f}_i^\top\mathbf{\Lambda}^\top\mathbf{\Psi}^{-1}\mathbf{\Lambda}\boldsymbol{f}_i -\frac{1}{2}\sum_{i=1}^n\boldsymbol{f}_i^\top\boldsymbol{f}_i\\

=& -\frac{(n+k)p}{2}\log(2\pi)-\frac{n}{2} \sum_{j=1}^p\log\psi_{jj}-\frac{1}{2}\sum_{i=1}^n\sum_{j}^p \frac{y_{ij}^2}{\psi_{jj}} +   \sum_{i=1}^n \boldsymbol{y}_i^\top\mathbf{\Psi}^{-1}\mathbf{\Lambda}\boldsymbol{f}_i  - \frac{1}{2}\sum_{i=1}^n\text{tr}\left(
\mathbf{\Lambda}^\top\mathbf{\Psi}^{-1}\mathbf{\Lambda}\boldsymbol{f}_i\boldsymbol{f}_i^\top\right) -\frac{1}{2}\sum_{i=1}^n\boldsymbol{f}_i^\top\boldsymbol{f}_i\\
\end{align*}

\$\$
\end{proof}

Now let us deduce the conditional expectation to \(\boldsymbol{f}\)
given \(\boldsymbol{y},\boldsymbol{\theta}_{(t)}\), denoted as
\(\mathbb{E}\left(\ell(\boldsymbol{\theta}_{(t)})\right)\). In
Equation~\ref{eq-loglikelihood}, the first term is independent of
\(\boldsymbol{f}\), hence stay the same under conditional expectation.
The last term is independent of \(\boldsymbol{\theta}\), therefore we
can regard it as a constant in
\(\mathbb{E}\left(\ell(\boldsymbol{\theta}_{(t)})\right)\).

Without ambiguity, denote \(\mathbb{E}[f_{ik}|_{(t)}]\) to be the
conditional expectation
\(\mathbb{E}[f_{ik}|\boldsymbol{y},\boldsymbol{\theta}_{(t)}]\) for
simplification. Then the conditional expectation
\(\mathbb{E}\left(\ell(\boldsymbol{\theta}_{(t)})\right)\) is given by

\[
\begin{align*}
\mathbb{E}\left(\ell(\boldsymbol{\theta}_{(t)})|_{(t)}\right)=& \ \text{constant}-\frac{n}{2}\sum_{j=1}^p\log{\psi_{jj}}-\\
&\frac{1}{2}\sum_{i=1}^n\sum_{j=1}^p \frac{\mathbb{E}\left[(y_{ij} - \sum_{k=1}^m \lambda_{jk}f_{ik})^2|_{(t)}\right]}{\psi_{jj}} - \frac{1}{2}\sum_{i=1}^n\mathbb{E}(\boldsymbol{f}_i^\top\boldsymbol{f}_i|_{(t)})\\
\end{align*}
\]

To deal with \(\mathbb{E}[\boldsymbol{f}_i|_{(t)}]\) and
\(\mathbb{E}[\boldsymbol{f}_i\boldsymbol{f}_i^\top|_{(t)}]\), we only
need to know the mean and variance of conditional distribution
\(\boldsymbol{f}_i|\boldsymbol{Y},\boldsymbol{\Lambda}^\top_{(t)},\boldsymbol{\Psi}_{(t)}\),
or equivalently
\(\boldsymbol{f}_i|\boldsymbol{y}_i,\boldsymbol{\Lambda}^\top_{(t)},\boldsymbol{\Psi}_{(t)}\)
because of the independency of \(\boldsymbol{f}_i\) to
\(\boldsymbol{y}_j\) for \(i\neq j\). This is because we can always
treat \(\mathbb{E}[\boldsymbol{f}_i\boldsymbol{f}_i^\top|_{(t)}]\) as

\[
\mathbb{E}[\boldsymbol{f}_i\boldsymbol{f}_i^\top|_{(t)}]= \mathbb{V}[\boldsymbol{f}_i|_{(t)}]+ \mathbb{E}[\boldsymbol{f}_i|_{(t)}]\mathbb{E}[\boldsymbol{f}_i|_{(t)}]^\top.
\] where \(\mathbb{V}[\boldsymbol{f}_i|_{(t)}]\) is the variance of
conditional distribution. To deal with this, wed need the following
lemma.

In our scenario, using Lemma~\ref{lem-cond}, we have

\[
\begin{align*}
& \boldsymbol{\mu}_{\boldsymbol{f}_i}= \boldsymbol{0}_k, \boldsymbol{\Sigma}_{\boldsymbol{f}_{i}}=\boldsymbol{I}_k\\
& \boldsymbol{\mu}_{\boldsymbol{y}_i}= \boldsymbol{0}_p, \boldsymbol{\Sigma}_{\boldsymbol{y}_i}=\boldsymbol{\Lambda}\boldsymbol{\Lambda}^\top+\boldsymbol{\Psi}
\end{align*} 
\]

and also

\[
\text{Cov}(\boldsymbol{y}_i,\boldsymbol{f}_i)=\text{Cov}(\boldsymbol{\Lambda f}_i+\boldsymbol{\epsilon}_i,\boldsymbol{f}_i)=\boldsymbol{\Lambda}^\top.
\]

Therefore we have

\[
\begin{align*}
&\mathbb{E}(\boldsymbol{f}_i|_{(t)})=\boldsymbol{\mu}_{\boldsymbol{f}_i|\boldsymbol{y}_i}=\boldsymbol{\Lambda}_{(t)}^\top(\boldsymbol{\Lambda}_{(t)}\boldsymbol{\Lambda}_{(t)}^\top+\boldsymbol{\Psi}_{(t)})^{-1}\boldsymbol{y}_i\\
&\mathbb{V}(\boldsymbol{f}_i|_{(t)})=\boldsymbol{\Sigma}_{\boldsymbol{f}_i|\boldsymbol{y}_i}=\boldsymbol{I}_k-\boldsymbol{\Lambda}_{(t)}^\top(\boldsymbol{\Lambda}_{(t)}\boldsymbol{\Lambda}_{(t)}^\top+\boldsymbol{\Psi}_{(t)})^{-1}\boldsymbol{\Lambda}_{(t)}.\\
\end{align*} 
\]

For simplification, let us denote

\[
\begin{align*}
&\boldsymbol{A}:=\boldsymbol{\Lambda}_{(t)}^\top(\boldsymbol{\Lambda}_{(t)}\boldsymbol{\Lambda}_{(t)}^\top+\boldsymbol{\Psi}_{(t)})^{-1}\\
&\boldsymbol{B}:=\boldsymbol{I}_k-\boldsymbol{\Lambda}_{(t)}^\top(\boldsymbol{\Lambda}_{(t)}\boldsymbol{\Lambda}_{(t)}^\top+\boldsymbol{\Psi}_{(t)})^{-1}\boldsymbol{\Lambda}_{(t)},\\
\end{align*} 
\]

we will get

\[
\begin{align*}
&\mathbb{E}(\boldsymbol{f}_i|_{(t)})= \boldsymbol{A}\boldsymbol{y}_i\\
&\mathbb{E}(\boldsymbol{f}_i\boldsymbol{f}_i^\top|_{(t)})= \boldsymbol{B}+\boldsymbol{A}\boldsymbol{y}_i\boldsymbol{y}_i^\top\boldsymbol{A}^\top.
\end{align*}
\]

Our expectation will finally be confirmed by

\[
\begin{align*}
El_{(t)}(\boldsymbol{\Lambda},\boldsymbol{\Psi})= & -\frac{n}{2}\sum_{j=1}^p\log{\Psi_{jj}}\\
& -\frac{1}{2}\sum_{i=1}^n\sum_{j=1}^p\frac{y_{ij}^2-2y_{ij}\boldsymbol{\Lambda}(j,)\boldsymbol{A}\boldsymbol{y}_i}{\Psi_{jj}}\\
& -\frac{1}{2}\sum_{i=1}^n\sum_{j=1}^p\frac{\boldsymbol{\Lambda}(j,)(\boldsymbol{B}+\boldsymbol{A}\boldsymbol{y}_i\boldsymbol{y}_i^\top\boldsymbol{A}^\top)\boldsymbol{\Lambda}(j,)^\top}{\Psi_{jj}}+\text{constant}.
\end{align*}
\]

\subsubsection{M-step}\label{m-step}

In M-step, we need to maximize so called \(Q\)-function with respect to
parameters where \(Q\)-function is penalized conditional expectation of
the log-likelihood, i.e. \[
Q(\boldsymbol{\Lambda},\boldsymbol{\Psi}) = El_{(t)}(\boldsymbol{\Lambda},\boldsymbol{\Psi}) - \frac{1}{2}P_{\rho}(\boldsymbol{\Lambda})
\] We add a coefficient \(\frac{1}{2}\) before
\(P_{\rho}(\boldsymbol{\Lambda})\) for simplification since we notice
that the same coefficient occurs in each term of conditional expectation
\(El_{(t)}(\boldsymbol{\Lambda},\boldsymbol{\Psi})\). When execute
M-step, we use the following strategy\citep{Ng2012EMAlgorithm}:

\begin{enumerate}
\def\labelenumi{\arabic{enumi}.}
\tightlist
\item
  Find \(\boldsymbol{\Psi}_{(k+1)}\) using current
  \(\boldsymbol{\Lambda}_{(t)}\), i.e. \[
    \boldsymbol{\Psi}_{(k+1)} = \arg \max_{\boldsymbol{\Psi}} (Q(\boldsymbol{\Lambda},\boldsymbol{\Psi})|\boldsymbol{\Lambda}=\boldsymbol{\Lambda}_{(t)}).
  \]
\item
  Find \(\boldsymbol{\Lambda}_{(k+1)}\) using
  \(\boldsymbol{\Psi}_{(k+1)}\) we got in previous step, i.e.~ \[
    \boldsymbol{\Lambda}_{(k+1)} = \arg \max_{\boldsymbol{\Lambda}} (Q(\boldsymbol{\Lambda},\boldsymbol{\Psi})|\boldsymbol{\Psi}=\boldsymbol{\Psi}_{(k+1)}).
  \]
\end{enumerate}

For step 1, take partial derivative with respect to each \(\psi_{jj}\)
and let it equal to zero to find the local maximizer of \(\psi_{jj}\),
i.e.~

\[
\frac{\partial}{\partial \psi_{jj}}Q(\boldsymbol{\Lambda},\boldsymbol{\Psi})|\boldsymbol{\Lambda}=\boldsymbol{\Lambda}_{(t)})=0.
\] By simple calculation, we will have

\[
\begin{align*}
\frac{\partial}{\partial \psi_{jj}}Q((\boldsymbol{\Lambda},\boldsymbol{\Psi})|\boldsymbol{\Lambda}=\boldsymbol{\Lambda}_{(t)})=
&-\frac{n}{2}\frac{1}{\psi_{jj}}+\frac{1}{2}\sum_{i=1}^n
\frac{y_{ij}^2-2y_{ij}\boldsymbol{\Lambda}_{(t)}(j,)\boldsymbol{A}\boldsymbol{y}_i}{\psi_{jj}^2} \\
&+ \frac{1}{2}\sum_{i=1}^n\frac{\boldsymbol{\Lambda}_{(t)}(j,)(\boldsymbol{B}+\boldsymbol{A}\boldsymbol{y}_i\boldsymbol{y}_i^\top\boldsymbol{A}^\top)\boldsymbol{\Lambda}_{(t)}(j,)^\top}{\psi_{jj}^2}.\\
\end{align*}
\]

Thus the update of \(\boldsymbol{\Psi}\) will be elementwisely given by

\[
\begin{equation}
\psi_{jj,(t+1)}=\frac{1}{n}\sum_{i=1}^n y_{ij}^2-\frac{2}{n}\sum_{i=1}^ny_{ij}\boldsymbol{\Lambda}_{(t)}(j,)\boldsymbol{A}\boldsymbol{y}_i + \frac{1}{n}\sum_{i=1}^n \boldsymbol{\Lambda}_{(t)}(j,)(\boldsymbol{B}+\boldsymbol{A}\boldsymbol{y}_i\boldsymbol{y}_i^\top\boldsymbol{A}^\top)\boldsymbol{\Lambda}_{(t)}(j,)^\top.
\end{equation}
\] But notice that the \(Q\)-function is not concave globally, therefore
we may update \(\boldsymbol{\Psi}\) selectively. More specifically, we
only update \(\psi_{jj}\) when it satisfies \[
\frac{\partial^2}{\partial\psi_{jj}^2}Q((\boldsymbol{\Lambda},\boldsymbol{\Psi})|\boldsymbol{\Lambda}=\boldsymbol{\Lambda}_{(t)})\leq 0,
\] i.e. \[
\psi_{jj}\leq \frac{2}{n}[\sum_{i=1}^n(y_{ij}^2-2y_{ij}\boldsymbol{\Lambda}_{(t)}(j,)\boldsymbol{A}\boldsymbol{y}_i)+\sum_{i=1}^n\boldsymbol{\Lambda}_{(t)}(j,)(\boldsymbol{B}+\boldsymbol{A}\boldsymbol{y}_i\boldsymbol{y}_i^\top\boldsymbol{A}^\top)\boldsymbol{\Lambda}_{(t)}(j,)^\top].
\] For step 2, let us revise the update formula, we have

\[
\begin{align*}
\boldsymbol{\Lambda}_{(k+1)}=&\arg \max_{\boldsymbol{\Lambda}} (Q(\boldsymbol{\Lambda,\boldsymbol{\Psi}})|\boldsymbol{\Psi}=\boldsymbol{\Psi}_{(k+1)})\\
=&\arg \max_{\boldsymbol{\Lambda}} \{\frac{1}{2} \sum_{i=1}^n \sum_{j=1}^p \frac{2y_{ij}\boldsymbol{\Lambda}(j,)\boldsymbol{A}\boldsymbol{y}_i}{\psi_{jj,(k+1)}}\\
&-\frac{1}{2}\sum_{i=1}^n\sum_{j=1}^p\frac{\boldsymbol{\Lambda}(j,)(\boldsymbol{B}+\boldsymbol{A}\boldsymbol{y}_i\boldsymbol{y}_i^\top\boldsymbol{A}^\top)\boldsymbol{\Lambda}(j,)^\top}{\psi_{jj,(k+1)}}-\frac{1}{2}P_{\rho}(\boldsymbol{\Lambda})\\
&-\frac{n}{2}\sum_{j=1}^p\log{\psi_{jj,(k+1)}}-\frac{1}{2}\sum_{i=1}^n\sum_{j=1}^p\frac{y_{ij}^2}{\psi_{jj,(k+1)}}+\text{constant}\}.\\
\end{align*}
\]

Since the last three terms do not contain any \(\boldsymbol{\Lambda}\),
so they can be eliminated. After letting
\(P_\rho(\boldsymbol{\Lambda}):=\rho\sum_{j=1}^p\sum_{i=1}^k|\lambda_{ji}|=\rho \sum_{j=1}^p||\boldsymbol{\Lambda}(j,)^\top||_1\)
as LASSO, we can rewrite it as

\$\$ \begin{align*}
\boldsymbol{\Lambda}_{(k+1)}

=& \arg \min_{\boldsymbol{\Lambda}}\{-\sum_{i=1}^n \sum_{j=1}^p \frac{2y_{ij}\boldsymbol{\Lambda}(j,)\boldsymbol{A}\boldsymbol{y}_i}{\psi_{jj,(k+1)}}\\

& + \sum_{i=1}^n\sum_{j=1}^p\frac{\boldsymbol{\Lambda}(j,)(\boldsymbol{B}+\boldsymbol{A}\boldsymbol{y}_i\boldsymbol{y}_i^\top\boldsymbol{A}^\top)\boldsymbol{\Lambda}(j,)^\top}{\psi_{jj,(k+1)}} + \rho \sum_{j=1}^p||\boldsymbol{\Lambda}(j,)^\top||_1
  \}
\end{align*} \$\$

Notice that the objective function with respect to
\(\boldsymbol{\Lambda}(q,)\) for any given \(q=1,2,\dots,p\) is convex,
and due to non-differentiablity of the \(L_1\)-norm at certain points, a
proximal gradient method can be necessary to optimize
\(\boldsymbol{\Lambda}\) row by row.

Denote the differentiable part (w.r.t. the \(q\)-th row of
\(\boldsymbol{\Lambda}\)) of the objective function as \[
G(\boldsymbol{\Lambda}):= \sum_{i=1}^n\sum_{j=1}^p\frac{\boldsymbol{\Lambda}(j,)(\boldsymbol{B}+\boldsymbol{A}\boldsymbol{y}_i\boldsymbol{y}_i^\top\boldsymbol{A}^\top)\boldsymbol{\Lambda}(j,)^\top}{\psi_{jj,(k+1)}} -\sum_{i=1}^n \sum_{j=1}^p \frac{2y_{ij}\boldsymbol{\Lambda}(j,)\boldsymbol{A}\boldsymbol{y}_i}{\psi_{jj,(k+1)}},
\] while the non-differentiable part is denoted as \[
H(\boldsymbol{\Lambda}) := \rho \sum_{j=1}^p||\boldsymbol{\Lambda}(j,)^\top||_1
\] We will have a nice form for the objective function, under which a
proximal gradient method can be applied for rowwisely upgrade the
loading matrix.

\begin{lemma}[]\protect\hypertarget{lem-proximalmethod}{}\label{lem-proximalmethod}

Consider the optimization problem \[
\min_{\boldsymbol{x}\in\mathbb{R}^n}\{G(\boldsymbol{x})+H(\boldsymbol{x})\},
\] where

\begin{enumerate}
\def\labelenumi{\arabic{enumi}.}
\tightlist
\item
  \(G(\boldsymbol{x}): \mathbb{R}^n \to \mathbb{R}\) is convex,
  differentiable, and \(L\)-Lipschitz continuous,
\item
  \(H(\boldsymbol{x}): \mathbb{R}^n \to (-\infty,\infty]\) is proper,
  lower semi-continuous and convex.
\end{enumerate}

The proximal gradient method is that pick an initial guess
\(\boldsymbol{x}_0\), for \(k=0,1,2,\dots\), repeat \[
\boldsymbol{\xi}_k := \boldsymbol{x}_k - s_k \nabla G(\boldsymbol{x})
\] \[
\boldsymbol{x}_{k+1} := \arg \min_{\boldsymbol{x}\in\mathbb{R}^n}\{H(\boldsymbol{x})+\frac{1}{2s_k}||\boldsymbol{x}-\boldsymbol{\xi}_k||^2_2\}
\] with a properly chosen step size \(0<s_k<\frac{1}{L}\).

\end{lemma}

One should notice that in practice, we may choose a sufficiently small
constant step size \(s_k=s\). In our case, the strategy to update the
loading matrix is

\begin{tcolorbox}[enhanced jigsaw, toptitle=1mm, arc=.35mm, colframe=quarto-callout-note-color-frame, leftrule=.75mm, colbacktitle=quarto-callout-note-color!10!white, bottomtitle=1mm, breakable, toprule=.15mm, opacitybacktitle=0.6, coltitle=black, bottomrule=.15mm, titlerule=0mm, title=\textcolor{quarto-callout-note-color}{\faInfo}\hspace{0.5em}{Algorithm}, colback=white, rightrule=.15mm, left=2mm, opacityback=0]

For each row of the \(\boldsymbol{\Lambda}\), say,
\(\boldsymbol{\Lambda}(q,)\) where \(q=1,2,\dots,p\), initialize it as
the updated row of loading matrix in the last M-step respectively. Then
for \(k=1,2,\dots,\), update the \(q\)-th row of
\(\boldsymbol{\Lambda}\) by repeating \[
\boldsymbol{\xi}_k := \boldsymbol{\Lambda}(q,)_k - s_k \nabla G(\boldsymbol{\Lambda}(q,)_k)
\] \[
\boldsymbol{\Lambda}(q,)_{k+1}:=\text{sign}(\boldsymbol{\xi}_k) \cdot \max(|\boldsymbol{\xi}_k|-\rho,0)
\] where \(|\cdot|\) is the elementwise absolute value and
\(\nabla G(\boldsymbol{\Lambda}(q,))\) can be calculated by \[
\nabla G(\boldsymbol{\Lambda}(q,)) = \sum_{i=1}^n\frac{(2\boldsymbol{B}+2\boldsymbol{A}\boldsymbol{y}_i\boldsymbol{y}_i^\top\boldsymbol{A}^\top)\boldsymbol{\Lambda}(q,)^\top}{\psi_{qq,(k+1)}}-\sum_{i=1}^n\frac{2y_{iq}\boldsymbol{A}\boldsymbol{y}_i}{\psi_{qq,(k+1)}}.
\]

\end{tcolorbox}

\subsection{Overall Algorithm for Fitting LASSO-penalized
FA}\label{overall-algorithm-for-fitting-lasso-penalized-fa}

\begin{tcolorbox}[enhanced jigsaw, toptitle=1mm, arc=.35mm, colframe=quarto-callout-note-color-frame, leftrule=.75mm, colbacktitle=quarto-callout-note-color!10!white, bottomtitle=1mm, breakable, toprule=.15mm, opacitybacktitle=0.6, coltitle=black, bottomrule=.15mm, titlerule=0mm, title=\textcolor{quarto-callout-note-color}{\faInfo}\hspace{0.5em}{Algorithm}, colback=white, rightrule=.15mm, left=2mm, opacityback=0]

\end{tcolorbox}

\section{Existing R packages}\label{existing-r-packages}

\section{Selecting initial values}\label{selecting-initial-values}

\begin{itemize}
\tightlist
\item
  MLE approach (non-penalised approach)
\item
  MLE + Rotation
\item
  FA model -\textgreater{} one order at a time
\item
  uniform/fixed
\end{itemize}

\section{Simulation setting}\label{simulation-setting}

Criteria? Judge based on estimation? Sparsity structure.

\subsection{Setting 1}\label{setting-1}

\begin{itemize}
\tightlist
\item
  Hirose + other existing simulation settings in published papers
\item
  Different FA order (k = 2, 4, 6, 8 x n = 200, 400, 800, 1600 x sparse
  vs non-sparse).
\end{itemize}

\subsection{Setting 2}\label{setting-2}

\bookmarksetup{startatroot}

\chapter{Results}\label{sec-results}

\bookmarksetup{startatroot}

\chapter{Appendix}\label{appendix}

\section{Multivariate Normal
distribution}\label{multivariate-normal-distribution}

Suppose that a \(p\)-vector random variable
\(\boldsymbol{y} \sim N_p(\boldsymbol{\mu},\mathbf{\Sigma})\), then the
probability density function of \(\boldsymbol{y}\) is given as:

\[f(\boldsymbol{y}) = (2\pi)^{-\frac{p}{2}}|\boldsymbol{\Sigma}|^{-\frac{1}{2}}\exp\left(-\frac{1}{2}(\boldsymbol{y}-\boldsymbol{\mu})^\top\boldsymbol{\Sigma}^{-1}(\boldsymbol{y}-\boldsymbol{\mu})\right).\]

\[\log f(\boldsymbol{y}) = -\frac{p}{2}\log(2\pi)-\frac{1}{2}\log|\boldsymbol{\Sigma}|-\frac{1}{2}(\boldsymbol{y}-\boldsymbol{\mu})^\top\boldsymbol{\Sigma}^{-1}(\boldsymbol{y}-\boldsymbol{\mu}).\]

\section{Conditional}\label{conditional}

\begin{lemma}[]\protect\hypertarget{lem-cond}{}\label{lem-cond}

If
\(\boldsymbol{\alpha}\sim N(\boldsymbol{\mu}_{\boldsymbol{\alpha}},\boldsymbol{\Sigma_{\boldsymbol{\alpha}}})\)
and
\(\boldsymbol{\beta}\sim N(\boldsymbol{\mu}_{\boldsymbol{\beta}},\boldsymbol{\Sigma}_{\boldsymbol{\beta}})\),
then we have
\(\boldsymbol{\alpha}|\boldsymbol{\beta}\sim N(\boldsymbol{\mu}_{\boldsymbol{\alpha}|\boldsymbol{\beta}},\boldsymbol{\Sigma}_{\boldsymbol{\alpha}|\boldsymbol{\beta}})\),
where \[
\begin{align*}
& \boldsymbol{\mu}_{\boldsymbol{\alpha}}|\boldsymbol{\beta}=\boldsymbol{\mu}_{\boldsymbol{\alpha}}+\boldsymbol{Cov}[\boldsymbol{\alpha},\boldsymbol{\beta}]\boldsymbol{\Sigma}_{\boldsymbol{\beta}}^{-1}\boldsymbol{\beta}\\
& \boldsymbol{\Sigma}_{\boldsymbol{\alpha}|\boldsymbol{\beta}}= \boldsymbol{\Sigma}_{\boldsymbol{\alpha}}-\boldsymbol{Cov}[\boldsymbol{\alpha},\boldsymbol{\beta}]\boldsymbol{\Sigma}_{\boldsymbol{\beta}}^{-1}\boldsymbol{Cov}[\boldsymbol{\alpha},\boldsymbol{\beta}]^\top.
\end{align*}
\]

\end{lemma}

\begin{lemma}[]\protect\hypertarget{lem-con-distribution}{}\label{lem-con-distribution}

Let
\(\boldsymbol{x}\sim \mathcal{N}(\boldsymbol{\mu}_1,\boldsymbol{\Sigma}_1)\)
and
\(\boldsymbol{y}\sim \mathcal{N}(\boldsymbol{\mu}_2,\boldsymbol{\Sigma}_2)\),
by definition of multivariate normal distribution, we have \[
\begin{pmatrix}
\boldsymbol{x}\\
\boldsymbol{y}
\end{pmatrix}
\sim \mathcal{N}(
\begin{pmatrix}
\boldsymbol{\mu_x}\\
\boldsymbol{\mu_y}
\end{pmatrix}
,
\begin{pmatrix}
\boldsymbol{\Sigma_{xx}} & \boldsymbol{\Sigma_{xy}}\\
\boldsymbol{\Sigma_{yx}} & \boldsymbol{\Sigma_{yy}}
\end{pmatrix}
).
\] The conditional distribution \(\boldsymbol{x}|\boldsymbol{y}\) is
then given by \[
\boldsymbol{x}|\boldsymbol{y} \sim \mathcal{N}(\boldsymbol{\mu_{x|y}},\boldsymbol{\Sigma_{x|y}}),
\] where \[
\begin{align*}
& \boldsymbol{\mu_{x|y}} = \boldsymbol{\mu_x}+\boldsymbol{\Sigma_{xy}}\boldsymbol{\Sigma_{yy}}^{-1}(\boldsymbol{y}-\boldsymbol{\mu_y})\\
& \boldsymbol{\Sigma_{x|y}} = \boldsymbol{\Sigma_{xx}}-\boldsymbol{\Sigma_{xy}}\boldsymbol{\Sigma_{yy}}^{-1}\boldsymbol{\Sigma_{yx}}
\end{align*}.
\]

\end{lemma}

\begin{proof}
See Appendix A.
\end{proof}

\bookmarksetup{startatroot}

\chapter*{References}\label{references}
\addcontentsline{toc}{chapter}{References}

\markboth{References}{References}

\renewcommand{\bibsection}{}
\bibliography{ref.bib}


\backmatter



\end{document}
