% Options for packages loaded elsewhere
\PassOptionsToPackage{unicode}{hyperref}
\PassOptionsToPackage{hyphens}{url}
\PassOptionsToPackage{dvipsnames,svgnames,x11names}{xcolor}
%
\documentclass[
  12pt,
  letterpaper,
  DIV=11,
  numbers=noendperiod]{scrartcl}

\usepackage{amsmath,amssymb}
\usepackage{iftex}
\ifPDFTeX
  \usepackage[T1]{fontenc}
  \usepackage[utf8]{inputenc}
  \usepackage{textcomp} % provide euro and other symbols
\else % if luatex or xetex
  \usepackage{unicode-math}
  \defaultfontfeatures{Scale=MatchLowercase}
  \defaultfontfeatures[\rmfamily]{Ligatures=TeX,Scale=1}
\fi
\usepackage{lmodern}
\ifPDFTeX\else  
    % xetex/luatex font selection
\fi
% Use upquote if available, for straight quotes in verbatim environments
\IfFileExists{upquote.sty}{\usepackage{upquote}}{}
\IfFileExists{microtype.sty}{% use microtype if available
  \usepackage[]{microtype}
  \UseMicrotypeSet[protrusion]{basicmath} % disable protrusion for tt fonts
}{}
\makeatletter
\@ifundefined{KOMAClassName}{% if non-KOMA class
  \IfFileExists{parskip.sty}{%
    \usepackage{parskip}
  }{% else
    \setlength{\parindent}{0pt}
    \setlength{\parskip}{6pt plus 2pt minus 1pt}}
}{% if KOMA class
  \KOMAoptions{parskip=half}}
\makeatother
\usepackage{xcolor}
\setlength{\emergencystretch}{3em} % prevent overfull lines
\setcounter{secnumdepth}{-\maxdimen} % remove section numbering
% Make \paragraph and \subparagraph free-standing
\ifx\paragraph\undefined\else
  \let\oldparagraph\paragraph
  \renewcommand{\paragraph}[1]{\oldparagraph{#1}\mbox{}}
\fi
\ifx\subparagraph\undefined\else
  \let\oldsubparagraph\subparagraph
  \renewcommand{\subparagraph}[1]{\oldsubparagraph{#1}\mbox{}}
\fi


\providecommand{\tightlist}{%
  \setlength{\itemsep}{0pt}\setlength{\parskip}{0pt}}\usepackage{longtable,booktabs,array}
\usepackage{calc} % for calculating minipage widths
% Correct order of tables after \paragraph or \subparagraph
\usepackage{etoolbox}
\makeatletter
\patchcmd\longtable{\par}{\if@noskipsec\mbox{}\fi\par}{}{}
\makeatother
% Allow footnotes in longtable head/foot
\IfFileExists{footnotehyper.sty}{\usepackage{footnotehyper}}{\usepackage{footnote}}
\makesavenoteenv{longtable}
\usepackage{graphicx}
\makeatletter
\def\maxwidth{\ifdim\Gin@nat@width>\linewidth\linewidth\else\Gin@nat@width\fi}
\def\maxheight{\ifdim\Gin@nat@height>\textheight\textheight\else\Gin@nat@height\fi}
\makeatother
% Scale images if necessary, so that they will not overflow the page
% margins by default, and it is still possible to overwrite the defaults
% using explicit options in \includegraphics[width, height, ...]{}
\setkeys{Gin}{width=\maxwidth,height=\maxheight,keepaspectratio}
% Set default figure placement to htbp
\makeatletter
\def\fps@figure{htbp}
\makeatother

\usepackage{fullpage}
\usepackage{enumitem}
\KOMAoption{captions}{tableheading}
\makeatletter
\@ifpackageloaded{caption}{}{\usepackage{caption}}
\AtBeginDocument{%
\ifdefined\contentsname
  \renewcommand*\contentsname{Table of contents}
\else
  \newcommand\contentsname{Table of contents}
\fi
\ifdefined\listfigurename
  \renewcommand*\listfigurename{List of Figures}
\else
  \newcommand\listfigurename{List of Figures}
\fi
\ifdefined\listtablename
  \renewcommand*\listtablename{List of Tables}
\else
  \newcommand\listtablename{List of Tables}
\fi
\ifdefined\figurename
  \renewcommand*\figurename{Figure}
\else
  \newcommand\figurename{Figure}
\fi
\ifdefined\tablename
  \renewcommand*\tablename{Table}
\else
  \newcommand\tablename{Table}
\fi
}
\@ifpackageloaded{float}{}{\usepackage{float}}
\floatstyle{ruled}
\@ifundefined{c@chapter}{\newfloat{codelisting}{h}{lop}}{\newfloat{codelisting}{h}{lop}[chapter]}
\floatname{codelisting}{Listing}
\newcommand*\listoflistings{\listof{codelisting}{List of Listings}}
\makeatother
\makeatletter
\makeatother
\makeatletter
\@ifpackageloaded{caption}{}{\usepackage{caption}}
\@ifpackageloaded{subcaption}{}{\usepackage{subcaption}}
\makeatother
\ifLuaTeX
  \usepackage{selnolig}  % disable illegal ligatures
\fi
\usepackage{bookmark}

\IfFileExists{xurl.sty}{\usepackage{xurl}}{} % add URL line breaks if available
\urlstyle{same} % disable monospaced font for URLs
\hypersetup{
  colorlinks=true,
  linkcolor={blue},
  filecolor={Maroon},
  citecolor={Blue},
  urlcolor={Blue},
  pdfcreator={LaTeX via pandoc}}

\author{}
\date{2024-04-17}

\begin{document}

\section{The EM Algorithm}\label{sec-EM}

\subsection{A brief introduction about the EM
algorithm}\label{a-brief-introduction-about-the-em-algorithm}

The Expectation-Maximization (EM) algorithm is a widely used iterative
method to compute the maximum likelihood estimation, especially when we
have some unobserved data(\textbf{Ng2012EMAlgorithm?}). As we mentioned,
the key of maximum likelihood estimation is solving equation
\begin{equation}
\frac{\partial}{\partial \beta}l=0.
\end{equation} However, challenges often arise from the complex nature
of the log-likelihood function, especially with data that is grouped,
censored, or truncated. To navigate these difficulties, the EM algorithm
introduces an ingenious approach by conceptualizing an equivalent
statistical problem that incorporates both observed and unobserved data.
Here, \emph{augmented data}(or complete) refers to the integration of
this unobserved component, enhancing the algorithm's ability to
iteratively estimate through two distinct phases: the Expectation step
(E-step) and the Maximization step (M-step). The iteration between these
steps facilitates the efficient of parameter estimates, making the EM
algorithm an essential tool for handling incomplete datasets
effectively.

\subsection{The E-step and M-step}\label{the-e-step-and-m-step}

Let \(\boldsymbol{x}\) denote the vector containing complete data,
\(\boldsymbol{y}\) denote the observed incomplete data and
\(\boldsymbol{z}\) denote the vector containing the missing data. Here
`missing data' is not necessarily missing, even if it does not at first
appear to be missed, we can formulating it to be as such to facilitate
the computation (we may see this later)(\textbf{Ng2012EMAlgorithm?}).

Now denote \(g_c(\boldsymbol{x};\boldsymbol{\Psi})\) as the probability
density function (p.d.f.) of the random vector \(\boldsymbol{X}\)
corresponding to \(\boldsymbol{x}\). Then the complete-data
log-likelihood function when complete data is fully observed can be
given by
\[\log L_c(\boldsymbol{\Psi})=\log g_c(\boldsymbol{x};\boldsymbol{\Psi}).\]
The EM algorithm approaches the problem of solving the incomplete-data
likelihood equation indirectly by proceeding iteratively in terms of
\(\log L_c(\boldsymbol{\Psi})\). But it is unobservable since it
includes missing part of the data, then we use the conditional
expectation given \(\boldsymbol{y}\) and current fit for
\(\boldsymbol{\Psi}\).

On the \((k+1)^th\) iteration, we have \begin{align*}
&\text{E-step: Compute } Q(\boldsymbol{\Psi};\boldsymbol{\Psi}^{(k)}):=\mathbb{E}[\log L_c(\boldsymbol{\Psi})|y]\\
&\text{M-step: Update }\boldsymbol{\Psi}^{(k+1)} \text{ as }\boldsymbol{{\Psi}}^{(k+1)}:=\text{arg}\max_\boldsymbol{\Psi} Q(\boldsymbol{\Psi};\boldsymbol{\Psi}^{(k)})
\end{align*}



\end{document}
